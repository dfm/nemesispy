%% Generated by Sphinx.
\def\sphinxdocclass{report}
\documentclass[letterpaper,10pt,english]{sphinxmanual}
\ifdefined\pdfpxdimen
   \let\sphinxpxdimen\pdfpxdimen\else\newdimen\sphinxpxdimen
\fi \sphinxpxdimen=.75bp\relax
\ifdefined\pdfimageresolution
    \pdfimageresolution= \numexpr \dimexpr1in\relax/\sphinxpxdimen\relax
\fi
%% let collapsible pdf bookmarks panel have high depth per default
\PassOptionsToPackage{bookmarksdepth=5}{hyperref}

\PassOptionsToPackage{warn}{textcomp}
\usepackage[utf8]{inputenc}
\ifdefined\DeclareUnicodeCharacter
% support both utf8 and utf8x syntaxes
  \ifdefined\DeclareUnicodeCharacterAsOptional
    \def\sphinxDUC#1{\DeclareUnicodeCharacter{"#1}}
  \else
    \let\sphinxDUC\DeclareUnicodeCharacter
  \fi
  \sphinxDUC{00A0}{\nobreakspace}
  \sphinxDUC{2500}{\sphinxunichar{2500}}
  \sphinxDUC{2502}{\sphinxunichar{2502}}
  \sphinxDUC{2514}{\sphinxunichar{2514}}
  \sphinxDUC{251C}{\sphinxunichar{251C}}
  \sphinxDUC{2572}{\textbackslash}
\fi
\usepackage{cmap}
\usepackage[T1]{fontenc}
\usepackage{amsmath,amssymb,amstext}
\usepackage{babel}



\usepackage{tgtermes}
\usepackage{tgheros}
\renewcommand{\ttdefault}{txtt}



\usepackage[Bjarne]{fncychap}
\usepackage{sphinx}

\fvset{fontsize=auto}
\usepackage{geometry}


% Include hyperref last.
\usepackage{hyperref}
% Fix anchor placement for figures with captions.
\usepackage{hypcap}% it must be loaded after hyperref.
% Set up styles of URL: it should be placed after hyperref.
\urlstyle{same}

\addto\captionsenglish{\renewcommand{\contentsname}{Contents:}}

\usepackage{sphinxmessages}
\setcounter{tocdepth}{1}



\title{NemesisPy}
\date{Mar 10, 2024}
\release{0.1}
\author{Jingxuan Yang, Juan Alday, Patrick Irwin}
\newcommand{\sphinxlogo}{\vbox{}}
\renewcommand{\releasename}{Release}
\makeindex
\begin{document}

\ifdefined\shorthandoff
  \ifnum\catcode`\=\string=\active\shorthandoff{=}\fi
  \ifnum\catcode`\"=\active\shorthandoff{"}\fi
\fi

\pagestyle{empty}
\sphinxmaketitle
\pagestyle{plain}
\sphinxtableofcontents
\pagestyle{normal}
\phantomsection\label{\detokenize{index::doc}}


\sphinxAtStartPar
\sphinxstylestrong{NemesisPy} contains routines for calculating and fitting
exoplanet emission spectra at arbitrary orbital phases,
which can help us constrain the thermal structure and chemical
abundance of exoplanet atmospheres. It is also capable
of fitting emission spectra at multiple orbital phases
(phase curves) at the same time. This package
comes ready with some spectral data and General Circulation
Model (GCM) data so you can start simulating spectra immediately.
There are a few demonstration routines in
the \sphinxtitleref{nemesispy/examples} folder; in particular, \sphinxtitleref{demo\_fit\_eclipse.py}
contains an interactive plot routine which allows you
to fit a hot Jupiter eclipse spectrum by hand by varying
its chemical abundance and temperature profile. This package
can be easily integrated with a Bayesian sampler such as
\sphinxtitleref{MultiNest} for a full spectral retrieval.

\sphinxAtStartPar
The radiative transfer calculations are done with the
correlated\sphinxhyphen{}k approximation, and are accelerated with the
\sphinxtitleref{Numba} just\sphinxhyphen{}in\sphinxhyphen{}time compiler to match the speed of
compiled languages such as Fortran. The radiative transfer
routines are based on the well\sphinxhyphen{}tested Nemesis (\sphinxurl{https://github.com/nemesiscode})
library developed by Patrick Irwin (University of Oxford) and collaborators.

\sphinxAtStartPar
This package has the following features:
\begin{itemize}
\item {} 
\sphinxAtStartPar
Written fully in Python: highly portable and customisable compared
to packages written in compiled languages and
can be easily installed on computer clusters.

\item {} 
\sphinxAtStartPar
Fast calculation speed: the most time consuming routines are optimised with
just\sphinxhyphen{}in\sphinxhyphen{}time compilation, which compiles Python code to machine
code at run time.

\item {} 
\sphinxAtStartPar
Radiative transfer routines are benchmarked against
the extensively used Nemesis (\sphinxurl{https://github.com/nemesiscode}) library.

\item {} 
\sphinxAtStartPar
Contains interactive plotting routines that allow you
to visualise the impact of gas abundance and thermal
structure on the emission spectra.

\item {} 
\sphinxAtStartPar
Contains routines to simulate spectra from General
Circulation Models (GCMs).

\item {} 
\sphinxAtStartPar
Contains unit tests to check if
the code is working correctly after modifications.

\end{itemize}

\begin{sphinxVerbatim}[commandchars=\\\{\}]
\PYG{g+gp}{\PYGZdl{} }pip\PYG{+w}{ }install\PYG{+w}{ }\PYGZhy{}\PYGZhy{}editable\PYG{+w}{ }.
\end{sphinxVerbatim}

\sphinxAtStartPar
To run all unit tests, change directory to the software folder and type the
following in the terminal:
.. code\sphinxhyphen{}block:: console
\begin{quote}

\sphinxAtStartPar
\$ python \sphinxhyphen{}m unittest discover test/
\end{quote}

\begin{sphinxadmonition}{note}{Note:}
\sphinxAtStartPar
This project is under active development.
\end{sphinxadmonition}


\chapter{Contents}
\label{\detokenize{index:contents}}
\sphinxstepscope


\section{Usage}
\label{\detokenize{usage:usage}}\label{\detokenize{usage::doc}}

\subsection{Installation}
\label{\detokenize{usage:installation}}
\sphinxAtStartPar
To use NemesisPy, first clone the GitHub repository (\sphinxurl{https://github.com/Jingxuan97/nemesispy})
to your computer. Then, navigate to the directory where you have saved the
repository and run the command

\begin{sphinxVerbatim}[commandchars=\\\{\}]
\PYG{g+gp}{\PYGZdl{} }pip\PYG{+w}{ }install\PYG{+w}{ }.
\end{sphinxVerbatim}

\sphinxAtStartPar
This will install the package and make it available to use in your Python environment.
In order to install the package but still make it editable, run instead the command

\begin{sphinxVerbatim}[commandchars=\\\{\}]
\PYG{g+gp}{\PYGZdl{} }pip\PYG{+w}{ }install\PYG{+w}{ }.\PYG{+w}{ }\PYGZhy{}\PYGZhy{}editable
\end{sphinxVerbatim}

\sphinxAtStartPar
NemesisPy can also by direcly installed from PyPI
(\sphinxurl{https://pypi.org/project/nemesispy/}) by running the “pip install nemesispy” command.
However, we strongly recommend installing the package from the GitHub repository
to make sure that you have the latest version of the package.

\sphinxstepscope


\section{API}
\label{\detokenize{api:api}}\label{\detokenize{api::doc}}
\sphinxAtStartPar
Here are the some of the functions that are available in the \sphinxtitleref{NemesisPy} package.


\subsection{common}
\label{\detokenize{api:common}}\index{calc\_hydrostat() (in module nemesispy)@\spxentry{calc\_hydrostat()}\spxextra{in module nemesispy}}

\begin{fulllineitems}
\phantomsection\label{\detokenize{api:nemesispy.calc_hydrostat}}
\pysigstartsignatures
\pysiglinewithargsret{\sphinxcode{\sphinxupquote{nemesispy.}}\sphinxbfcode{\sphinxupquote{calc\_hydrostat}}}{\emph{\DUrole{n}{P}}, \emph{\DUrole{n}{T}}, \emph{\DUrole{n}{mmw}}, \emph{\DUrole{n}{M\_plt}}, \emph{\DUrole{n}{R\_plt}}, \emph{\DUrole{n}{H}\DUrole{o}{=}\DUrole{default_value}{array({[}{]}, dtype=float64)}}}{}
\pysigstopsignatures
\sphinxAtStartPar
Calculates an altitude profile from given pressure, temperature and
mean molecular weight profiles assuming hydrostatic equilibrium.
\begin{quote}\begin{description}
\sphinxlineitem{Parameters}\begin{itemize}
\item {} 
\sphinxAtStartPar
\sphinxstyleliteralstrong{\sphinxupquote{P}} (\sphinxstyleliteralemphasis{\sphinxupquote{ndarray}}) \textendash{} Pressure profile
Unit: Pa

\item {} 
\sphinxAtStartPar
\sphinxstyleliteralstrong{\sphinxupquote{T}} (\sphinxstyleliteralemphasis{\sphinxupquote{ndarray}}) \textendash{} Temperature profile
Unit: K

\item {} 
\sphinxAtStartPar
\sphinxstyleliteralstrong{\sphinxupquote{mmw}} (\sphinxstyleliteralemphasis{\sphinxupquote{ndarray}}) \textendash{} Mean molecular weight profile
Unit: kg

\item {} 
\sphinxAtStartPar
\sphinxstyleliteralstrong{\sphinxupquote{M\_plt}} (\sphinxstyleliteralemphasis{\sphinxupquote{real}}) \textendash{} Planetary mass
Unit: kg

\item {} 
\sphinxAtStartPar
\sphinxstyleliteralstrong{\sphinxupquote{R\_plt}} (\sphinxstyleliteralemphasis{\sphinxupquote{real}}) \textendash{} Planetary radius
Unit: m

\item {} 
\sphinxAtStartPar
\sphinxstyleliteralstrong{\sphinxupquote{H}} (\sphinxstyleliteralemphasis{\sphinxupquote{ndarray}}) \textendash{} Altitude profile to be adjusted
Unit: m

\end{itemize}

\sphinxlineitem{Returns}
\sphinxAtStartPar
\sphinxstylestrong{adjusted\_H} \textendash{} Altitude profile satisfying hydrostatic equlibrium.
Unit: m

\sphinxlineitem{Return type}
\sphinxAtStartPar
ndarray

\end{description}\end{quote}

\end{fulllineitems}

\index{disc\_weights() (in module nemesispy)@\spxentry{disc\_weights()}\spxextra{in module nemesispy}}

\begin{fulllineitems}
\phantomsection\label{\detokenize{api:nemesispy.disc_weights}}
\pysigstartsignatures
\pysiglinewithargsret{\sphinxcode{\sphinxupquote{nemesispy.}}\sphinxbfcode{\sphinxupquote{disc\_weights}}}{\emph{\DUrole{n}{n}}}{}
\pysigstopsignatures
\sphinxAtStartPar
Generate weights for disc integration in the the emission angle
direction.
\begin{quote}\begin{description}
\sphinxlineitem{Parameters}
\sphinxAtStartPar
\sphinxstyleliteralstrong{\sphinxupquote{n}} (\sphinxstyleliteralemphasis{\sphinxupquote{int}}) \textendash{} Number of emission angles. Minuim 2.

\sphinxlineitem{Returns}
\sphinxAtStartPar
\begin{itemize}
\item {} 
\sphinxAtStartPar
\sphinxstylestrong{mu} (\sphinxstyleemphasis{ndarray}) \textendash{} List of cos(emission angle) for integration.

\item {} 
\sphinxAtStartPar
\sphinxstylestrong{wtmu} (\sphinxstyleemphasis{ndarray}) \textendash{} List of weights for integration.

\end{itemize}


\end{description}\end{quote}

\end{fulllineitems}

\index{gauss\_lobatto\_weights() (in module nemesispy)@\spxentry{gauss\_lobatto\_weights()}\spxextra{in module nemesispy}}

\begin{fulllineitems}
\phantomsection\label{\detokenize{api:nemesispy.gauss_lobatto_weights}}
\pysigstartsignatures
\pysiglinewithargsret{\sphinxcode{\sphinxupquote{nemesispy.}}\sphinxbfcode{\sphinxupquote{gauss\_lobatto\_weights}}}{\emph{\DUrole{n}{phase}}, \emph{\DUrole{n}{nmu}}}{}
\pysigstopsignatures
\sphinxAtStartPar
Given the orbital phase, calculates the coordinates and weights of the points
on a disc needed to compute the disc integrated radiance.

\sphinxAtStartPar
The points are chosen on a number of rings according to Gauss\sphinxhyphen{}Lobatto quadrature
scheme, and spaced on the rings according to trapezium rule.

\sphinxAtStartPar
Refer to the begining of the trig.py file for geomety and convections.
\begin{quote}\begin{description}
\sphinxlineitem{Parameters}\begin{itemize}
\item {} 
\sphinxAtStartPar
\sphinxstyleliteralstrong{\sphinxupquote{phase}} (\sphinxstyleliteralemphasis{\sphinxupquote{real}}) \textendash{} Stellar phase/orbital phase in degrees.
0=parimary transit and increase to 180 at secondary eclipse.

\item {} 
\sphinxAtStartPar
\sphinxstyleliteralstrong{\sphinxupquote{nmu}} (\sphinxstyleliteralemphasis{\sphinxupquote{integer}}) \textendash{} Number of zenith angle ordinates

\end{itemize}

\sphinxlineitem{Returns}
\sphinxAtStartPar
\begin{itemize}
\item {} 
\sphinxAtStartPar
\sphinxstylestrong{nav} (\sphinxstyleemphasis{int}) \textendash{} Number of FOV points

\item {} 
\sphinxAtStartPar
\sphinxstylestrong{wav} (\sphinxstyleemphasis{ndarray}) \textendash{} FOV\sphinxhyphen{}averaging table:
0th row is lattitude, 1st row is longitude, 2nd row is stellar zenith
angle, 3rd row is emission zenith angle, 4th row is stellar azimuth angle,
5th row is weight.

\end{itemize}


\end{description}\end{quote}

\end{fulllineitems}

\index{get\_gas\_name() (in module nemesispy)@\spxentry{get\_gas\_name()}\spxextra{in module nemesispy}}

\begin{fulllineitems}
\phantomsection\label{\detokenize{api:nemesispy.get_gas_name}}
\pysigstartsignatures
\pysiglinewithargsret{\sphinxcode{\sphinxupquote{nemesispy.}}\sphinxbfcode{\sphinxupquote{get\_gas\_name}}}{\emph{\DUrole{n}{id}}}{}
\pysigstopsignatures
\sphinxAtStartPar
Find the name of the molecule given its ID number.
\begin{quote}\begin{description}
\sphinxlineitem{Parameters}
\sphinxAtStartPar
\sphinxstyleliteralstrong{\sphinxupquote{id}} (\sphinxstyleliteralemphasis{\sphinxupquote{int}}) \textendash{} ID of the molecule

\sphinxlineitem{Returns}
\sphinxAtStartPar
\sphinxstylestrong{name} \textendash{} Name of the molecule.

\sphinxlineitem{Return type}
\sphinxAtStartPar
str

\end{description}\end{quote}

\end{fulllineitems}

\index{get\_gas\_id() (in module nemesispy)@\spxentry{get\_gas\_id()}\spxextra{in module nemesispy}}

\begin{fulllineitems}
\phantomsection\label{\detokenize{api:nemesispy.get_gas_id}}
\pysigstartsignatures
\pysiglinewithargsret{\sphinxcode{\sphinxupquote{nemesispy.}}\sphinxbfcode{\sphinxupquote{get\_gas\_id}}}{\emph{\DUrole{n}{name}}}{}
\pysigstopsignatures
\sphinxAtStartPar
Find the ID of the molecule given its name.
\begin{quote}\begin{description}
\sphinxlineitem{Parameters}
\sphinxAtStartPar
\sphinxstyleliteralstrong{\sphinxupquote{name}} (\sphinxstyleliteralemphasis{\sphinxupquote{str}}) \textendash{} Name of the molecule

\sphinxlineitem{Returns}
\sphinxAtStartPar
\sphinxstylestrong{id} \textendash{} ID of the molecule

\sphinxlineitem{Return type}
\sphinxAtStartPar
int

\end{description}\end{quote}

\end{fulllineitems}

\index{interp\_gcm\_X() (in module nemesispy)@\spxentry{interp\_gcm\_X()}\spxextra{in module nemesispy}}

\begin{fulllineitems}
\phantomsection\label{\detokenize{api:nemesispy.interp_gcm_X}}
\pysigstartsignatures
\pysiglinewithargsret{\sphinxcode{\sphinxupquote{nemesispy.}}\sphinxbfcode{\sphinxupquote{interp\_gcm\_X}}}{\emph{\DUrole{n}{lon}}, \emph{\DUrole{n}{lat}}, \emph{\DUrole{n}{p}}, \emph{\DUrole{n}{gcm\_lon}}, \emph{\DUrole{n}{gcm\_lat}}, \emph{\DUrole{n}{gcm\_p}}, \emph{\DUrole{n}{X}}, \emph{\DUrole{n}{substellar\_point\_longitude\_shift}\DUrole{o}{=}\DUrole{default_value}{0}}}{}
\pysigstopsignatures
\sphinxAtStartPar
Find the profile of X as a function of pressure at a location specified by
(lon,lat) by interpolating a GCM. X can be any scalar quantity modeled in
the GCM, for example temperature or chemical abundance.
Note: gcm\_lon and gcm\_lat must be strictly increasing.
\begin{quote}\begin{description}
\sphinxlineitem{Parameters}\begin{itemize}
\item {} 
\sphinxAtStartPar
\sphinxstyleliteralstrong{\sphinxupquote{lon}} (\sphinxstyleliteralemphasis{\sphinxupquote{real}}) \textendash{} Longitude of the location

\item {} 
\sphinxAtStartPar
\sphinxstyleliteralstrong{\sphinxupquote{lat}} (\sphinxstyleliteralemphasis{\sphinxupquote{real}}) \textendash{} Latitude of the location

\item {} 
\sphinxAtStartPar
\sphinxstyleliteralstrong{\sphinxupquote{p}} (\sphinxstyleliteralemphasis{\sphinxupquote{ndarray}}) \textendash{} Pressure grid for the output X(P) profile

\item {} 
\sphinxAtStartPar
\sphinxstyleliteralstrong{\sphinxupquote{gcm\_lon}} (\sphinxstyleliteralemphasis{\sphinxupquote{ndarray}}) \textendash{} Longitude grid of the GCM, assumed to be {[}\sphinxhyphen{}180,180{]} and increasing. (1D)

\item {} 
\sphinxAtStartPar
\sphinxstyleliteralstrong{\sphinxupquote{gcm\_lat}} (\sphinxstyleliteralemphasis{\sphinxupquote{ndarray}}) \textendash{} Latitude grid of the GCM, assumed to be {[}\sphinxhyphen{}90,90{]} and increasing. (1D)

\item {} 
\sphinxAtStartPar
\sphinxstyleliteralstrong{\sphinxupquote{gcm\_p}} (\sphinxstyleliteralemphasis{\sphinxupquote{ndarrray}}) \textendash{} Pressure grid of the GCM. (1D)

\item {} 
\sphinxAtStartPar
\sphinxstyleliteralstrong{\sphinxupquote{X}} (\sphinxstyleliteralemphasis{\sphinxupquote{ndarray}}) \textendash{} A scalar quantity defined in the GCM, e.g., temperature or VMR. (3D)
Has dimensioin NLON x NLAT x NP

\item {} 
\sphinxAtStartPar
\sphinxstyleliteralstrong{\sphinxupquote{substellar\_point\_longitude\_shift}} (\sphinxstyleliteralemphasis{\sphinxupquote{real}}) \textendash{} The longitude shift from the output coordinate system to the
coordinate system of the GCM. For example, if in the output
coordinate system the substellar point is defined at 0 E,
whereas in the GCM coordinate system the substellar point is defined
at 180 E, put substellar\_point\_longitude\_shift=180.

\end{itemize}

\sphinxlineitem{Returns}
\sphinxAtStartPar
\begin{itemize}
\item {} 
\sphinxAtStartPar
\sphinxstylestrong{interped\_X} (\sphinxstyleemphasis{ndarray}) \textendash{} X interpolated to (lon,lat,p).

\item {} 
\sphinxAtStartPar
\sphinxstyleemphasis{Important}

\item {} 
\sphinxAtStartPar
\sphinxstyleemphasis{Longitudinal values outside of the gcm longtidinal grid are interpolated}

\item {} 
\sphinxAtStartPar
\sphinxstyleemphasis{properly using the periodicity of longitude. However, latitudinal values}

\item {} 
\sphinxAtStartPar
\sphinxstyleemphasis{outside of the gcm latitude grid are interpolated using the boundary}

\item {} 
\sphinxAtStartPar
\sphinxstyleemphasis{values of the gcm grid. In practice, this reduces the accuracy of}

\item {} 
\sphinxAtStartPar
\sphinxstyleemphasis{interpolation in the polar regions outside of the gcm grid; in particular,}

\item {} 
\sphinxAtStartPar
\sphinxstyleemphasis{the interpolated value at the poles will be dependent on longitude.}

\item {} 
\sphinxAtStartPar
\sphinxstyleemphasis{This is a negligible source of error in disc integrated spectroscopy since}

\item {} 
\sphinxAtStartPar
\sphinxstyleemphasis{the contribution of radiance is weighted by cos(latitude).}

\end{itemize}


\end{description}\end{quote}

\end{fulllineitems}



\subsection{models}
\label{\detokenize{api:models}}\index{TP\_Guillot() (in module nemesispy)@\spxentry{TP\_Guillot()}\spxextra{in module nemesispy}}

\begin{fulllineitems}
\phantomsection\label{\detokenize{api:nemesispy.TP_Guillot}}
\pysigstartsignatures
\pysiglinewithargsret{\sphinxcode{\sphinxupquote{nemesispy.}}\sphinxbfcode{\sphinxupquote{TP\_Guillot}}}{\emph{\DUrole{n}{P}}, \emph{\DUrole{n}{g\_plt}}, \emph{\DUrole{n}{T\_eq}}, \emph{\DUrole{n}{k\_IR}}, \emph{\DUrole{n}{gamma}}, \emph{\DUrole{n}{f}}, \emph{\DUrole{n}{T\_int}\DUrole{o}{=}\DUrole{default_value}{100}}}{}
\pysigstopsignatures
\sphinxAtStartPar
TP profile from eqn. (29) in Guillot 2010.
DOI: 10.1051/0004\sphinxhyphen{}6361/200913396
Model parameters (4) : k\_IR, gamma, f, T\_int
\begin{quote}\begin{description}
\sphinxlineitem{Parameters}\begin{itemize}
\item {} 
\sphinxAtStartPar
\sphinxstyleliteralstrong{\sphinxupquote{P}} (\sphinxstyleliteralemphasis{\sphinxupquote{ndarray}}) \textendash{} Pressure grid (in Pa) on which the TP profile is to be constructed.

\item {} 
\sphinxAtStartPar
\sphinxstyleliteralstrong{\sphinxupquote{g\_plt}} (\sphinxstyleliteralemphasis{\sphinxupquote{real}}) \textendash{} Gravitational acceleration at the highest pressure in the pressure
grid.

\item {} 
\sphinxAtStartPar
\sphinxstyleliteralstrong{\sphinxupquote{T\_eq}} (\sphinxstyleliteralemphasis{\sphinxupquote{real}}) \textendash{} Temperature corresponding to the stellar flux.
T\_eq = T\_star * (R\_star/(2*semi\_major\_axis))**0.5

\item {} 
\sphinxAtStartPar
\sphinxstyleliteralstrong{\sphinxupquote{k\_IR}} (\sphinxstyleliteralemphasis{\sphinxupquote{real}}) \textendash{} Range {[}1e\sphinxhyphen{}5,1e3{]}
Mean absorption coefficient in the thermal wavelengths.

\item {} 
\sphinxAtStartPar
\sphinxstyleliteralstrong{\sphinxupquote{gamma}} (\sphinxstyleliteralemphasis{\sphinxupquote{real}}) \textendash{} Range \textasciitilde{} {[}1e\sphinxhyphen{}3,1e2{]}
gamma = k\_V/k\_IR, ratio of visible to thermal opacities

\item {} 
\sphinxAtStartPar
\sphinxstyleliteralstrong{\sphinxupquote{f}} (\sphinxstyleliteralemphasis{\sphinxupquote{real}}) \textendash{} f parameter (positive), See eqn. (29) in Guillot 2010.
With f = 1 at the substellar point, f = 1/2 for a
day\sphinxhyphen{}side average and f = 1/4 for whole planet surface average.

\item {} 
\sphinxAtStartPar
\sphinxstyleliteralstrong{\sphinxupquote{T\_int}} (\sphinxstyleliteralemphasis{\sphinxupquote{real}}) \textendash{} Temperature corresponding to the intrinsic heat flux of the planet.

\end{itemize}

\sphinxlineitem{Returns}
\sphinxAtStartPar
\sphinxstylestrong{TP} \textendash{} Temperature as a function of pressure.

\sphinxlineitem{Return type}
\sphinxAtStartPar
ndarray

\end{description}\end{quote}

\end{fulllineitems}

\index{TP\_Guillot14() (in module nemesispy)@\spxentry{TP\_Guillot14()}\spxextra{in module nemesispy}}

\begin{fulllineitems}
\phantomsection\label{\detokenize{api:nemesispy.TP_Guillot14}}
\pysigstartsignatures
\pysiglinewithargsret{\sphinxcode{\sphinxupquote{nemesispy.}}\sphinxbfcode{\sphinxupquote{TP\_Guillot14}}}{\emph{\DUrole{n}{P}}, \emph{\DUrole{n}{g\_plt}}, \emph{\DUrole{n}{T\_eq}}, \emph{\DUrole{n}{k\_IR}}, \emph{\DUrole{n}{gamma1}}, \emph{\DUrole{n}{gamma2}}, \emph{\DUrole{n}{alpha}}, \emph{\DUrole{n}{beta}}, \emph{\DUrole{n}{T\_int}}}{}
\pysigstopsignatures
\sphinxAtStartPar
TP profile from eqn. (20) in Line et al. 2012.
doi:10.1088/0004\sphinxhyphen{}637X/749/1/93, based on Parmentier and Guillot 2014.
10.1051/0004\sphinxhyphen{}6361/201322342.
Model parameters (5) : k\_IR, gamma1, gamma2, alpha, beta, T\_int
\begin{quote}\begin{description}
\sphinxlineitem{Parameters}\begin{itemize}
\item {} 
\sphinxAtStartPar
\sphinxstyleliteralstrong{\sphinxupquote{P}} (\sphinxstyleliteralemphasis{\sphinxupquote{ndarray}}) \textendash{} Pressure grid (in Pa) on which the TP profile is to be constructed.

\item {} 
\sphinxAtStartPar
\sphinxstyleliteralstrong{\sphinxupquote{g\_plt}} (\sphinxstyleliteralemphasis{\sphinxupquote{real}}) \textendash{} Gravitational acceleration at the highest pressure in the pressure
grid.

\item {} 
\sphinxAtStartPar
\sphinxstyleliteralstrong{\sphinxupquote{T\_eq}} (\sphinxstyleliteralemphasis{\sphinxupquote{real}}) \textendash{} Temperature corresponding to the stellar flux.
T\_eq = T\_star * (R\_star/(2*semi\_major\_axis))**0.5

\item {} 
\sphinxAtStartPar
\sphinxstyleliteralstrong{\sphinxupquote{k\_IR}} (\sphinxstyleliteralemphasis{\sphinxupquote{real}}) \textendash{} Range \textasciitilde{} {[}1e\sphinxhyphen{}5,1e3{]}
Mean absorption coefficient in the thermal wavelengths.
m\textasciicircum{}2/kg

\item {} 
\sphinxAtStartPar
\sphinxstyleliteralstrong{\sphinxupquote{gamma\_1}} (\sphinxstyleliteralemphasis{\sphinxupquote{real}}) \textendash{} Range \textasciitilde{} {[}1e\sphinxhyphen{}3,1e2{]}
gamma\_1 = k\_V1/k\_IR, ratio of mean opacity of the first visible stream
to mean opacity in the thermal stream.

\item {} 
\sphinxAtStartPar
\sphinxstyleliteralstrong{\sphinxupquote{gamma\_2}} (\sphinxstyleliteralemphasis{\sphinxupquote{real}}) \textendash{} Range \textasciitilde{} {[}1e\sphinxhyphen{}3,1e2{]}
gamma\_2 = k\_V2/k\_IR, ratio of mean opacity of the second visible stream
to mean opacity in the thermal stream.

\item {} 
\sphinxAtStartPar
\sphinxstyleliteralstrong{\sphinxupquote{alpha}} (\sphinxstyleliteralemphasis{\sphinxupquote{real}}) \textendash{} Range {[}0,1{]}
Percentage of the visible stream represented by opacity gamma1.

\item {} 
\sphinxAtStartPar
\sphinxstyleliteralstrong{\sphinxupquote{beta}} (\sphinxstyleliteralemphasis{\sphinxupquote{real}}) \textendash{} Range {[}0,2{]}
A catch all parameter for albedo, emissivity, day\sphinxhyphen{}night redistribution.

\item {} 
\sphinxAtStartPar
\sphinxstyleliteralstrong{\sphinxupquote{T\_int}} (\sphinxstyleliteralemphasis{\sphinxupquote{real}}) \textendash{} Temperature corresponding to the intrinsic heat flux of the planet.

\end{itemize}

\sphinxlineitem{Returns}
\sphinxAtStartPar
\sphinxstylestrong{TP} \textendash{} Temperature as a function of pressure.

\sphinxlineitem{Return type}
\sphinxAtStartPar
ndarray

\end{description}\end{quote}

\end{fulllineitems}

\index{Model4() (in module nemesispy)@\spxentry{Model4()}\spxextra{in module nemesispy}}

\begin{fulllineitems}
\phantomsection\label{\detokenize{api:nemesispy.Model4}}
\pysigstartsignatures
\pysiglinewithargsret{\sphinxcode{\sphinxupquote{nemesispy.}}\sphinxbfcode{\sphinxupquote{Model4}}}{\emph{\DUrole{n}{P\_grid}}, \emph{\DUrole{n}{lon\_grid}}, \emph{\DUrole{n}{lat\_grid}}, \emph{\DUrole{n}{g\_plt}}, \emph{\DUrole{n}{T\_eq}}, \emph{\DUrole{n}{scale}}, \emph{\DUrole{n}{phase\_offset}}, \emph{\DUrole{n}{log\_kappa\_day}}, \emph{\DUrole{n}{log\_gamma\_day}}, \emph{\DUrole{n}{log\_f\_day}}, \emph{\DUrole{n}{T\_int\_day}}, \emph{\DUrole{n}{log\_kappa\_night}}, \emph{\DUrole{n}{log\_gamma\_night}}, \emph{\DUrole{n}{log\_f\_night}}, \emph{\DUrole{n}{T\_int\_night}}, \emph{\DUrole{n}{n}}}{}
\pysigstopsignatures
\sphinxAtStartPar
2D temperature model consisting of two representative Guillot TP profiles.
See model 4 in Yang et al. 2023. (\sphinxurl{https://doi.org/10.1093/mnras/stad2555})
The atmosphere is partitioned (in longitude) into two regions: a dayside
and a nightside. The dayside longitudinal span is allowed to vary.
\begin{quote}\begin{description}
\sphinxlineitem{Parameters}\begin{itemize}
\item {} 
\sphinxAtStartPar
\sphinxstyleliteralstrong{\sphinxupquote{P\_grid}} (\sphinxstyleliteralemphasis{\sphinxupquote{ndarray}}) \textendash{} Pressure grid (in Pa) of the model.

\item {} 
\sphinxAtStartPar
\sphinxstyleliteralstrong{\sphinxupquote{lon\_grid}} (\sphinxstyleliteralemphasis{\sphinxupquote{ndarray}}) \textendash{} Longitude grid (in degree) of the model.
Substellar point is assumed to be at 0.
Range is {[}\sphinxhyphen{}180,180{]}.

\item {} 
\sphinxAtStartPar
\sphinxstyleliteralstrong{\sphinxupquote{lat\_grid}} (\sphinxstyleliteralemphasis{\sphinxupquote{ndarray}}) \textendash{} Latitude grid (in degree) of the model.
Range is {[}\sphinxhyphen{}90,90{]}.

\item {} 
\sphinxAtStartPar
\sphinxstyleliteralstrong{\sphinxupquote{g\_plt}} (\sphinxstyleliteralemphasis{\sphinxupquote{real}}) \textendash{} Gravitational acceleration at the highest pressure in the pressure
grid.

\item {} 
\sphinxAtStartPar
\sphinxstyleliteralstrong{\sphinxupquote{T\_eq}} (\sphinxstyleliteralemphasis{\sphinxupquote{real}}) \textendash{} Temperature corresponding to the stellar flux.
T\_eq = T\_star * (R\_star/(2*semi\_major\_axis))**0.5

\item {} 
\sphinxAtStartPar
\sphinxstyleliteralstrong{\sphinxupquote{scale}} (\sphinxstyleliteralemphasis{\sphinxupquote{real}}) \textendash{} Scaling parameter for the longitudinal span of the dayside.
Set to be between 0.5 and 1.2.

\item {} 
\sphinxAtStartPar
\sphinxstyleliteralstrong{\sphinxupquote{phase\_offset}} (\sphinxstyleliteralemphasis{\sphinxupquote{real}}) \textendash{} Central longitude of the dayside
Set to be between \sphinxhyphen{}45 and 45.

\item {} 
\sphinxAtStartPar
\sphinxstyleliteralstrong{\sphinxupquote{log\_kappa\_day}} (\sphinxstyleliteralemphasis{\sphinxupquote{real}}) \textendash{} Range {[}1e\sphinxhyphen{}5,1e3{]}
Mean absorption coefficient in the thermal wavelengths. (dayside)

\item {} 
\sphinxAtStartPar
\sphinxstyleliteralstrong{\sphinxupquote{log\_gamma\_day}} (\sphinxstyleliteralemphasis{\sphinxupquote{real}}) \textendash{} Range \textasciitilde{} {[}1e\sphinxhyphen{}3,1e2{]}
gamma = k\_V/k\_IR, ratio of visible to thermal opacities (dayside)

\item {} 
\sphinxAtStartPar
\sphinxstyleliteralstrong{\sphinxupquote{log\_f\_day}} (\sphinxstyleliteralemphasis{\sphinxupquote{real}}) \textendash{} f parameter (positive), See eqn. (29) in Guillot 2010.
With f = 1 at the substellar point, f = 1/2 for a
day\sphinxhyphen{}side average and f = 1/4 for whole planet surface average. (dayside)

\item {} 
\sphinxAtStartPar
\sphinxstyleliteralstrong{\sphinxupquote{T\_int\_day}} (\sphinxstyleliteralemphasis{\sphinxupquote{real}}) \textendash{} Temperature corresponding to the intrinsic heat flux of the planet.
(dayside)

\item {} 
\sphinxAtStartPar
\sphinxstyleliteralstrong{\sphinxupquote{log\_kappa\_night}} (\sphinxstyleliteralemphasis{\sphinxupquote{real}}) \textendash{} Same as above definitions but for nightside.

\item {} 
\sphinxAtStartPar
\sphinxstyleliteralstrong{\sphinxupquote{log\_gamma\_night}} (\sphinxstyleliteralemphasis{\sphinxupquote{real}}) \textendash{} Same as above definitions but for nightside.

\item {} 
\sphinxAtStartPar
\sphinxstyleliteralstrong{\sphinxupquote{log\_f\_night}} (\sphinxstyleliteralemphasis{\sphinxupquote{real}}) \textendash{} Same as above definitions but for nightside.

\item {} 
\sphinxAtStartPar
\sphinxstyleliteralstrong{\sphinxupquote{T\_int\_night}} (\sphinxstyleliteralemphasis{\sphinxupquote{real}}) \textendash{} Same as above definitions but for nightside.

\item {} 
\sphinxAtStartPar
\sphinxstyleliteralstrong{\sphinxupquote{n}} (\sphinxstyleliteralemphasis{\sphinxupquote{real}}) \textendash{} Parameter to control how temperature vary with longitude on the dayside.
Should be positive.

\end{itemize}

\sphinxlineitem{Returns}
\sphinxAtStartPar
\sphinxstylestrong{tp\_out} \textendash{} Temperature model defined on a (longitude, laitude, pressure) grid.

\sphinxlineitem{Return type}
\sphinxAtStartPar
ndarray

\end{description}\end{quote}

\end{fulllineitems}



\subsection{radtran}
\label{\detokenize{api:radtran}}\index{calc\_layer() (in module nemesispy)@\spxentry{calc\_layer()}\spxextra{in module nemesispy}}

\begin{fulllineitems}
\phantomsection\label{\detokenize{api:nemesispy.calc_layer}}
\pysigstartsignatures
\pysiglinewithargsret{\sphinxcode{\sphinxupquote{nemesispy.}}\sphinxbfcode{\sphinxupquote{calc\_layer}}}{\emph{\DUrole{n}{planet\_radius}}, \emph{\DUrole{n}{H\_model}}, \emph{\DUrole{n}{P\_model}}, \emph{\DUrole{n}{T\_model}}, \emph{\DUrole{n}{VMR\_model}}, \emph{\DUrole{n}{ID}}, \emph{\DUrole{n}{NLAYER}}, \emph{\DUrole{n}{path\_angle}}, \emph{\DUrole{n}{H\_0}\DUrole{o}{=}\DUrole{default_value}{0.0}}, \emph{\DUrole{n}{layer\_type}\DUrole{o}{=}\DUrole{default_value}{1}}, \emph{\DUrole{n}{custom\_path\_angle}\DUrole{o}{=}\DUrole{default_value}{0.0}}, \emph{\DUrole{n}{custom\_H\_base}\DUrole{o}{=}\DUrole{default_value}{None}}, \emph{\DUrole{n}{custom\_P\_base}\DUrole{o}{=}\DUrole{default_value}{None}}}{}
\pysigstopsignatures
\sphinxAtStartPar
Top level routine that calculates absorber\sphinxhyphen{}amount\sphinxhyphen{}weighted average
layer properties from an atmospehric model.
\begin{quote}\begin{description}
\sphinxlineitem{Parameters}\begin{itemize}
\item {} 
\sphinxAtStartPar
\sphinxstyleliteralstrong{\sphinxupquote{planet\_radius}} (\sphinxstyleliteralemphasis{\sphinxupquote{real}}) \textendash{} Reference planetary planet\_radius where H\_model is set to be 0.  Usually
set at surface for terrestrial planets, or at 1 bar pressure level for
gas giants.

\item {} 
\sphinxAtStartPar
\sphinxstyleliteralstrong{\sphinxupquote{H\_model}}\sphinxstyleliteralstrong{\sphinxupquote{(}}\sphinxstyleliteralstrong{\sphinxupquote{NMODEL}}\sphinxstyleliteralstrong{\sphinxupquote{)}} (\sphinxstyleliteralemphasis{\sphinxupquote{ndarray}}) \textendash{} Altitudes of the atmospheric model points.
Assumed to be increasing.
Unit: m

\item {} 
\sphinxAtStartPar
\sphinxstyleliteralstrong{\sphinxupquote{P\_model}}\sphinxstyleliteralstrong{\sphinxupquote{(}}\sphinxstyleliteralstrong{\sphinxupquote{NMODEL}}\sphinxstyleliteralstrong{\sphinxupquote{)}} (\sphinxstyleliteralemphasis{\sphinxupquote{ndarray}}) \textendash{} Pressures of the atmospheric model points.
Unit: Pa

\item {} 
\sphinxAtStartPar
\sphinxstyleliteralstrong{\sphinxupquote{T\_mode}}\sphinxstyleliteralstrong{\sphinxupquote{(}}\sphinxstyleliteralstrong{\sphinxupquote{NMODEL}}\sphinxstyleliteralstrong{\sphinxupquote{)}} (\sphinxstyleliteralemphasis{\sphinxupquote{ndarray}}) \textendash{} Temperature of the atmospheric model points.
Unit: K

\item {} 
\sphinxAtStartPar
\sphinxstyleliteralstrong{\sphinxupquote{VMR\_model}}\sphinxstyleliteralstrong{\sphinxupquote{(}}\sphinxstyleliteralstrong{\sphinxupquote{NMODEL}} (\sphinxstyleliteralemphasis{\sphinxupquote{ndarray}}) \textendash{} Volume mixing ratios of gases defined in the atmospheric model.
VMR\_model{[}i,j{]} is Volume Mixing Ratio of jth gas at ith profile point.
The jth column corresponds to the gas with RADTRANS ID ID{[}j{]}.

\item {} 
\sphinxAtStartPar
\sphinxstyleliteralstrong{\sphinxupquote{NGAS}}\sphinxstyleliteralstrong{\sphinxupquote{)}} (\sphinxstyleliteralemphasis{\sphinxupquote{ndarray}}) \textendash{} Volume mixing ratios of gases defined in the atmospheric model.
VMR\_model{[}i,j{]} is Volume Mixing Ratio of jth gas at ith profile point.
The jth column corresponds to the gas with RADTRANS ID ID{[}j{]}.

\item {} 
\sphinxAtStartPar
\sphinxstyleliteralstrong{\sphinxupquote{ID}} (\sphinxstyleliteralemphasis{\sphinxupquote{ndarray}}) \textendash{} Gas identifiers.

\item {} 
\sphinxAtStartPar
\sphinxstyleliteralstrong{\sphinxupquote{NLAYER}} (\sphinxstyleliteralemphasis{\sphinxupquote{int}}) \textendash{} Number of layers to split the atmospheric model into.

\item {} 
\sphinxAtStartPar
\sphinxstyleliteralstrong{\sphinxupquote{path\_angle}} (\sphinxstyleliteralemphasis{\sphinxupquote{real}}) \textendash{} Zenith angle in degrees defined at H\_0.

\item {} 
\sphinxAtStartPar
\sphinxstyleliteralstrong{\sphinxupquote{H\_0}} (\sphinxstyleliteralemphasis{\sphinxupquote{real}}\sphinxstyleliteralemphasis{\sphinxupquote{, }}\sphinxstyleliteralemphasis{\sphinxupquote{optional}}) \textendash{} Altitude of the lowest point in the atmospheric model.
This is defined with respect to the reference planetary radius, i.e.
the altitude at planet\_radius is 0.
The default is 0.0.

\item {} 
\sphinxAtStartPar
\sphinxstyleliteralstrong{\sphinxupquote{layer\_type}} (\sphinxstyleliteralemphasis{\sphinxupquote{int}}\sphinxstyleliteralemphasis{\sphinxupquote{, }}\sphinxstyleliteralemphasis{\sphinxupquote{optional}}) \textendash{} Integer specifying how to split up the layers.
0 = split by equal changes in pressure
1 = split by equal changes in log pressure
2 = split by equal changes in height
3 = split by equal changes in path length
4 = split by given layer base pressures P\_base
5 = split by given layer base heights H\_base
Note 4 and 5 force NLAYER = len(P\_base) or len(H\_base).
The default is 1.

\item {} 
\sphinxAtStartPar
\sphinxstyleliteralstrong{\sphinxupquote{custom\_path\_angle}} (\sphinxstyleliteralemphasis{\sphinxupquote{real}}\sphinxstyleliteralemphasis{\sphinxupquote{, }}\sphinxstyleliteralemphasis{\sphinxupquote{optional}}) \textendash{} Required only for layer type 3.
Zenith angle in degrees defined at the base of the lowest layer.
The default is 0.0.

\item {} 
\sphinxAtStartPar
\sphinxstyleliteralstrong{\sphinxupquote{custom\_H\_base}}\sphinxstyleliteralstrong{\sphinxupquote{(}}\sphinxstyleliteralstrong{\sphinxupquote{NLAYER}}\sphinxstyleliteralstrong{\sphinxupquote{)}} (\sphinxstyleliteralemphasis{\sphinxupquote{ndarray}}\sphinxstyleliteralemphasis{\sphinxupquote{, }}\sphinxstyleliteralemphasis{\sphinxupquote{optional}}) \textendash{} Required only for layer type 5.
Altitudes of the layer bases defined by user.
The default is None.

\item {} 
\sphinxAtStartPar
\sphinxstyleliteralstrong{\sphinxupquote{custom\_P\_base}}\sphinxstyleliteralstrong{\sphinxupquote{(}}\sphinxstyleliteralstrong{\sphinxupquote{NLAYER}}\sphinxstyleliteralstrong{\sphinxupquote{)}} (\sphinxstyleliteralemphasis{\sphinxupquote{ndarray}}\sphinxstyleliteralemphasis{\sphinxupquote{, }}\sphinxstyleliteralemphasis{\sphinxupquote{optional}}) \textendash{} Required only for layer type 4.
Pressures of the layer bases defined by user.
The default is None.

\end{itemize}

\sphinxlineitem{Returns}
\sphinxAtStartPar
\begin{itemize}
\item {} 
\sphinxAtStartPar
\sphinxstylestrong{H\_layer(NLAYER)} (\sphinxstyleemphasis{ndarray}) \textendash{} Averaged layer altitudes.
Unit: m

\item {} 
\sphinxAtStartPar
\sphinxstylestrong{P\_layer(NLAYER)} (\sphinxstyleemphasis{ndarray}) \textendash{} Averaged layer pressures.
Unit: Pa

\item {} 
\sphinxAtStartPar
\sphinxstylestrong{T\_layer(NLAYER)} (\sphinxstyleemphasis{ndarray}) \textendash{} Averaged layer temperatures.
Unit: K

\item {} 
\sphinxAtStartPar
\sphinxstylestrong{U\_layer(NLAYER)} (\sphinxstyleemphasis{ndarray}) \textendash{} Total gaseous absorber amounts along the line\sphinxhyphen{}of\sphinxhyphen{}sight path, i.e.
total number of gas molecules per unit area.
Unit: no of absorber per m\textasciicircum{}2

\item {} 
\sphinxAtStartPar
\sphinxstylestrong{VMR\_layer(NLAYER, NGAS)} (\sphinxstyleemphasis{ndarray}) \textendash{} Averaged layer volume mixing ratios.
VMR\_layer{[}i,j{]} is averaged VMR of gas j in layer i.

\item {} 
\sphinxAtStartPar
\sphinxstylestrong{scale(NLAYER)} (\sphinxstyleemphasis{ndarray}) \textendash{} Layer scaling factor, i.e. ratio of path length through each layer
to the layer thickness.

\item {} 
\sphinxAtStartPar
\sphinxstylestrong{dS(NLAYER)} (\sphinxstyleemphasis{ndarray}) \textendash{} Path lengths.
Unit: m

\end{itemize}


\end{description}\end{quote}

\end{fulllineitems}

\index{calc\_mmw() (in module nemesispy)@\spxentry{calc\_mmw()}\spxextra{in module nemesispy}}

\begin{fulllineitems}
\phantomsection\label{\detokenize{api:nemesispy.calc_mmw}}
\pysigstartsignatures
\pysiglinewithargsret{\sphinxcode{\sphinxupquote{nemesispy.}}\sphinxbfcode{\sphinxupquote{calc\_mmw}}}{\emph{\DUrole{n}{ID}}, \emph{\DUrole{n}{VMR}}, \emph{\DUrole{n}{ISO}\DUrole{o}{=}\DUrole{default_value}{{[}{]}}}}{}
\pysigstopsignatures
\sphinxAtStartPar
Calculate mean molecular weight in kg given a list of molecule IDs and
a list of their respective volume mixing ratios.
\begin{quote}\begin{description}
\sphinxlineitem{Parameters}\begin{itemize}
\item {} 
\sphinxAtStartPar
\sphinxstyleliteralstrong{\sphinxupquote{ID}} (\sphinxstyleliteralemphasis{\sphinxupquote{ndarray}}\sphinxstyleliteralemphasis{\sphinxupquote{ or }}\sphinxstyleliteralemphasis{\sphinxupquote{list}}) \textendash{} A list of Radtran gas identifiers.

\item {} 
\sphinxAtStartPar
\sphinxstyleliteralstrong{\sphinxupquote{VMR}} (\sphinxstyleliteralemphasis{\sphinxupquote{ndarray}}\sphinxstyleliteralemphasis{\sphinxupquote{ or }}\sphinxstyleliteralemphasis{\sphinxupquote{list}}) \textendash{} A list of VMRs corresponding to the gases in ID.

\item {} 
\sphinxAtStartPar
\sphinxstyleliteralstrong{\sphinxupquote{ISO}} (\sphinxstyleliteralemphasis{\sphinxupquote{ndarray}}\sphinxstyleliteralemphasis{\sphinxupquote{ or }}\sphinxstyleliteralemphasis{\sphinxupquote{list}}) \textendash{} If ISO={[}{]}, assume terrestrial relative isotopic abundance for all gases.
Otherwise, if ISO{[}i{]}=0, then use terrestrial relative isotopic abundance
for the ith gas. To specify particular isotopologue, input the
corresponding Radtran isotopologue identifiers.

\end{itemize}

\sphinxlineitem{Returns}
\sphinxAtStartPar
\sphinxstylestrong{mmw} \textendash{} Mean molecular weight.
Unit: kg

\sphinxlineitem{Return type}
\sphinxAtStartPar
real

\end{description}\end{quote}
\subsubsection*{Notes}

\sphinxAtStartPar
Cf mol\_id.py and mol\_info.py.

\end{fulllineitems}

\index{calc\_planck() (in module nemesispy)@\spxentry{calc\_planck()}\spxextra{in module nemesispy}}

\begin{fulllineitems}
\phantomsection\label{\detokenize{api:nemesispy.calc_planck}}
\pysigstartsignatures
\pysiglinewithargsret{\sphinxcode{\sphinxupquote{nemesispy.}}\sphinxbfcode{\sphinxupquote{calc\_planck}}}{\emph{\DUrole{n}{wave\_grid}}, \emph{\DUrole{n}{T}}, \emph{\DUrole{n}{ispace}\DUrole{o}{=}\DUrole{default_value}{1}}}{}
\pysigstopsignatures
\sphinxAtStartPar
Calculates the blackbody radiance.
\begin{quote}\begin{description}
\sphinxlineitem{Parameters}\begin{itemize}
\item {} 
\sphinxAtStartPar
\sphinxstyleliteralstrong{\sphinxupquote{wave\_grid}}\sphinxstyleliteralstrong{\sphinxupquote{(}}\sphinxstyleliteralstrong{\sphinxupquote{nwave}}\sphinxstyleliteralstrong{\sphinxupquote{)}} (\sphinxstyleliteralemphasis{\sphinxupquote{ndarray}}) \textendash{} Wavelength or wavenumber array
Unit: um or cm\sphinxhyphen{}1

\item {} 
\sphinxAtStartPar
\sphinxstyleliteralstrong{\sphinxupquote{T}} (\sphinxstyleliteralemphasis{\sphinxupquote{real}}) \textendash{} Temperature of the blackbody (K)

\item {} 
\sphinxAtStartPar
\sphinxstyleliteralstrong{\sphinxupquote{ispace}} (\sphinxstyleliteralemphasis{\sphinxupquote{int}}) \textendash{} Flag indicating the spectral units
(0) Wavenumber (cm\sphinxhyphen{}1)
(1) Wavelength (um)

\end{itemize}

\sphinxlineitem{Returns}
\sphinxAtStartPar
\begin{itemize}
\item {} 
\sphinxAtStartPar
\sphinxstylestrong{bb(nwave)} (\sphinxstyleemphasis{ndarray})

\item {} 
\sphinxAtStartPar
\sphinxstyleemphasis{Spectral radiance.}

\item {} 
\sphinxAtStartPar
\sphinxstylestrong{Unit} (\sphinxstyleemphasis{(0) W cm\sphinxhyphen{}2 sr\sphinxhyphen{}1 (cm\sphinxhyphen{}1)\sphinxhyphen{}1}) \textendash{}
\begin{enumerate}
\sphinxsetlistlabels{\arabic}{enumi}{enumii}{(}{)}%
\item {} 
\sphinxAtStartPar
W cm\sphinxhyphen{}2 sr\sphinxhyphen{}1 um\sphinxhyphen{}1

\end{enumerate}

\end{itemize}


\end{description}\end{quote}

\end{fulllineitems}

\index{calc\_radiance() (in module nemesispy)@\spxentry{calc\_radiance()}\spxextra{in module nemesispy}}

\begin{fulllineitems}
\phantomsection\label{\detokenize{api:nemesispy.calc_radiance}}
\pysigstartsignatures
\pysiglinewithargsret{\sphinxcode{\sphinxupquote{nemesispy.}}\sphinxbfcode{\sphinxupquote{calc\_radiance}}}{\emph{\DUrole{n}{wave\_grid}}, \emph{\DUrole{n}{U\_layer}}, \emph{\DUrole{n}{P\_layer}}, \emph{\DUrole{n}{T\_layer}}, \emph{\DUrole{n}{VMR\_layer}}, \emph{\DUrole{n}{k\_gas\_w\_g\_p\_t}}, \emph{\DUrole{n}{P\_grid}}, \emph{\DUrole{n}{T\_grid}}, \emph{\DUrole{n}{del\_g}}, \emph{\DUrole{n}{ScalingFactor}}, \emph{\DUrole{n}{R\_plt}}, \emph{\DUrole{n}{solspec}}, \emph{\DUrole{n}{k\_cia}}, \emph{\DUrole{n}{ID}}, \emph{\DUrole{n}{cia\_nu\_grid}}, \emph{\DUrole{n}{cia\_T\_grid}}, \emph{\DUrole{n}{dH}}}{}
\pysigstopsignatures
\sphinxAtStartPar
Calculate emission spectrum using the correlated\sphinxhyphen{}k method.
\begin{quote}\begin{description}
\sphinxlineitem{Parameters}\begin{itemize}
\item {} 
\sphinxAtStartPar
\sphinxstyleliteralstrong{\sphinxupquote{wave\_grid}}\sphinxstyleliteralstrong{\sphinxupquote{(}}\sphinxstyleliteralstrong{\sphinxupquote{NWAVE}}\sphinxstyleliteralstrong{\sphinxupquote{)}} (\sphinxstyleliteralemphasis{\sphinxupquote{ndarray}}) \textendash{} Wavelengths (um) grid for calculating spectra.

\item {} 
\sphinxAtStartPar
\sphinxstyleliteralstrong{\sphinxupquote{U\_layer}}\sphinxstyleliteralstrong{\sphinxupquote{(}}\sphinxstyleliteralstrong{\sphinxupquote{NLAYER}}\sphinxstyleliteralstrong{\sphinxupquote{)}} (\sphinxstyleliteralemphasis{\sphinxupquote{ndarray}}) \textendash{} Surface density of gas particles in each layer.
Unit: no. of particle/m\textasciicircum{}2

\item {} 
\sphinxAtStartPar
\sphinxstyleliteralstrong{\sphinxupquote{P\_layer}}\sphinxstyleliteralstrong{\sphinxupquote{(}}\sphinxstyleliteralstrong{\sphinxupquote{NLAYER}}\sphinxstyleliteralstrong{\sphinxupquote{)}} (\sphinxstyleliteralemphasis{\sphinxupquote{ndarray}}) \textendash{} Atmospheric pressure grid.
Unit: Pa

\item {} 
\sphinxAtStartPar
\sphinxstyleliteralstrong{\sphinxupquote{T\_layer}}\sphinxstyleliteralstrong{\sphinxupquote{(}}\sphinxstyleliteralstrong{\sphinxupquote{NLAYER}}\sphinxstyleliteralstrong{\sphinxupquote{)}} (\sphinxstyleliteralemphasis{\sphinxupquote{ndarray}}) \textendash{} Atmospheric temperature grid.
Unit: K

\item {} 
\sphinxAtStartPar
\sphinxstyleliteralstrong{\sphinxupquote{VMR\_layer}}\sphinxstyleliteralstrong{\sphinxupquote{(}}\sphinxstyleliteralstrong{\sphinxupquote{NLAYER}} (\sphinxstyleliteralemphasis{\sphinxupquote{ndarray}}) \textendash{} Array of volume mixing ratios for NGAS.
Has dimensioin: NLAYER x NGAS

\item {} 
\sphinxAtStartPar
\sphinxstyleliteralstrong{\sphinxupquote{NGAS}}\sphinxstyleliteralstrong{\sphinxupquote{)}} (\sphinxstyleliteralemphasis{\sphinxupquote{ndarray}}) \textendash{} Array of volume mixing ratios for NGAS.
Has dimensioin: NLAYER x NGAS

\item {} 
\sphinxAtStartPar
\sphinxstyleliteralstrong{\sphinxupquote{k\_gas\_w\_g\_p\_t}}\sphinxstyleliteralstrong{\sphinxupquote{(}}\sphinxstyleliteralstrong{\sphinxupquote{NGAS}} (\sphinxstyleliteralemphasis{\sphinxupquote{ndarray}}) \textendash{} k\sphinxhyphen{}coefficients.
Has dimension: NGAS x NWAVE x NG x NPRESSK x NTEMPK.

\item {} 
\sphinxAtStartPar
\sphinxstyleliteralstrong{\sphinxupquote{NWAVE}} (\sphinxstyleliteralemphasis{\sphinxupquote{ndarray}}) \textendash{} k\sphinxhyphen{}coefficients.
Has dimension: NGAS x NWAVE x NG x NPRESSK x NTEMPK.

\item {} 
\sphinxAtStartPar
\sphinxstyleliteralstrong{\sphinxupquote{NG}} (\sphinxstyleliteralemphasis{\sphinxupquote{ndarray}}) \textendash{} k\sphinxhyphen{}coefficients.
Has dimension: NGAS x NWAVE x NG x NPRESSK x NTEMPK.

\item {} 
\sphinxAtStartPar
\sphinxstyleliteralstrong{\sphinxupquote{NPRESSK}} (\sphinxstyleliteralemphasis{\sphinxupquote{ndarray}}) \textendash{} k\sphinxhyphen{}coefficients.
Has dimension: NGAS x NWAVE x NG x NPRESSK x NTEMPK.

\item {} 
\sphinxAtStartPar
\sphinxstyleliteralstrong{\sphinxupquote{NTEMPK}}\sphinxstyleliteralstrong{\sphinxupquote{)}} (\sphinxstyleliteralemphasis{\sphinxupquote{ndarray}}) \textendash{} k\sphinxhyphen{}coefficients.
Has dimension: NGAS x NWAVE x NG x NPRESSK x NTEMPK.

\item {} 
\sphinxAtStartPar
\sphinxstyleliteralstrong{\sphinxupquote{P\_grid}}\sphinxstyleliteralstrong{\sphinxupquote{(}}\sphinxstyleliteralstrong{\sphinxupquote{NPRESSK}}\sphinxstyleliteralstrong{\sphinxupquote{)}} (\sphinxstyleliteralemphasis{\sphinxupquote{ndarray}}) \textendash{} Pressure grid on which the k\sphinxhyphen{}coeff’s are pre\sphinxhyphen{}computed.
We want SI unit (Pa) here.

\item {} 
\sphinxAtStartPar
\sphinxstyleliteralstrong{\sphinxupquote{T\_grid}}\sphinxstyleliteralstrong{\sphinxupquote{(}}\sphinxstyleliteralstrong{\sphinxupquote{NTEMPK}}\sphinxstyleliteralstrong{\sphinxupquote{)}} (\sphinxstyleliteralemphasis{\sphinxupquote{ndarray}}) \textendash{} Temperature grid on which the k\sphinxhyphen{}coeffs are pre\sphinxhyphen{}computed. In Kelvin

\item {} 
\sphinxAtStartPar
\sphinxstyleliteralstrong{\sphinxupquote{del\_g}} (\sphinxstyleliteralemphasis{\sphinxupquote{ndarray}}) \textendash{} Quadrature weights of the g\sphinxhyphen{}ordinates.

\item {} 
\sphinxAtStartPar
\sphinxstyleliteralstrong{\sphinxupquote{ScalingFactor}}\sphinxstyleliteralstrong{\sphinxupquote{(}}\sphinxstyleliteralstrong{\sphinxupquote{NLAYER}}\sphinxstyleliteralstrong{\sphinxupquote{)}} (\sphinxstyleliteralemphasis{\sphinxupquote{ndarray}}) \textendash{} Scale stuff to line of sight

\item {} 
\sphinxAtStartPar
\sphinxstyleliteralstrong{\sphinxupquote{R\_plt}} (\sphinxstyleliteralemphasis{\sphinxupquote{real}}) \textendash{} Planetary radius
Unit: m

\item {} 
\sphinxAtStartPar
\sphinxstyleliteralstrong{\sphinxupquote{solspec}} (\sphinxstyleliteralemphasis{\sphinxupquote{ndarray}}) \textendash{} 
\sphinxAtStartPar
Stellar spectra, used when the unit of the output is in fraction
of stellar irradiance.

\sphinxAtStartPar
Stellar flux at planet’s distance (W cm\sphinxhyphen{}2 um\sphinxhyphen{}1 or W cm\sphinxhyphen{}2 (cm\sphinxhyphen{}1)\sphinxhyphen{}1)


\end{itemize}

\sphinxlineitem{Returns}
\sphinxAtStartPar
\sphinxstylestrong{spectrum} \textendash{} Output spectrum (W cm\sphinxhyphen{}2 um\sphinxhyphen{}1 sr\sphinxhyphen{}1)

\sphinxlineitem{Return type}
\sphinxAtStartPar
ndarray

\end{description}\end{quote}

\end{fulllineitems}




\renewcommand{\indexname}{Index}
\printindex
\end{document}